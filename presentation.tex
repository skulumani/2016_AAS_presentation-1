%\documentclass[11pt,professionalfonts,hyperref={pdftex,pdfpagemode=none,pdfstartview=FitH}]{beamer}
%\usepackage{times}
\documentclass[11pt,professionalfonts]{beamer}
\usefonttheme{serif}
\usepackage{presentation_packages}
\usepackage{multimedia}
\usepackage{animate} % for animated gif

\DeclareSIUnit\year{yr}

\newcommand{\vs}{\vspace{0.3cm}}

\definecolor{mygray}{gray}{0.9}
\definecolor{RoyalBlue}{rgb}{0.25,0.41,0.88}
\def\Emph{\textcolor{RoyalBlue}}

\definecolor{tmp}{rgb}{0.804,0.941,1.0}
\setbeamercolor{numerical}{fg=black,bg=tmp}
\setbeamercolor{exact}{fg=black,bg=red}

\mode<presentation> 
{
  \usetheme{Warsaw}
  \usefonttheme{serif}
  \setbeamercovered{transparent}
}

\setbeamertemplate{footline}%{split theme}
{%
  \leavevmode%
  \hbox{\begin{beamercolorbox}[wd=.5\paperwidth,ht=2.5ex,dp=1.125ex,leftskip=.3cm,rightskip=.3cm plus1fill]{author in head/foot}%
    \usebeamerfont{author in head/foot}\insertshorttitle
  \end{beamercolorbox}%
  \begin{beamercolorbox}[wd=.5\paperwidth,ht=2.5ex,dp=1.125ex,leftskip=.3cm,rightskip=.3cm]{title in head/foot}
%    \usebeamerfont{title in head/foot}\mypaper\hfill \insertframenumber/\inserttotalframenumber
    \usebeamerfont{title in head/foot}\hfill \insertframenumber/\inserttotalframenumber
  \end{beamercolorbox}}%
  \vskip0pt%
} \setbeamercolor{box}{fg=black,bg=yellow}


\title[Reachability Sets on \Poincare section]{\large\bf  Low-Thrust Trajectory Design Using Reachability Sets near Asteroid 4769 Castalia}

\author{\vspace*{-0.3cm}}

   
\institute{
	\footnotesize
	{\normalsize\bf{Shankar Kulumani and Taeyoung Lee}}\\
	\vspace*{0.2cm}
  	\textbf{Flight Dynamics \& Control Lab}\\ \vspace*{0.5cm}
 	\begin{figure} %figure%
       	\includegraphics[width=0.75\textwidth]{gw_txh_2cs_pos}
  	\end{figure}
}
\date{}

\begin{document}
%=======================================================%

\setcounter{framenumber}{-1}
\begin{frame} %-----------------------------%
  \titlepage
\end{frame}   %-----------------------------%

\section*{}
\subsection*{Introduction}  

\begin{frame}{Asteroid Missions}
\begin{itemize}
    \item Asteroid provide great insight into the early formation of the solar system
    \item Asteroid mining - extraction of useful materials from asteroids
    \item Large risk of future asteroid impacts - huge threat to humanity
\end{itemize}    

\begin{center}
    \includegraphics[height=0.3\textheight]{figures/near_mos_20001203_full.jpg}
    \hfill
    \includegraphics[height=0.3\textheight]{figures/Itokawa8_hayabusa_1210.jpg}
\end{center}


\end{frame}

\begin{frame}{Asteroid Mining}
    \begin{itemize}
      \item  Variety of useful materials can be extracted from asteroids to support:
      \begin{itemize}
          \item Propulsion, construction, life support, agriculture, and precious/strategic metals
      \end{itemize}
      \item Commercialization of near-Earth asteroids is possible
    \end{itemize}

\begin{center}
\small
    \begin{tabular}{|l|r|r|}
        \hline 
        Element & Price (\SI{}[US]{\$\per\kilo\gram}) & Sales (\SI{}{\$M\per\year}) \\
        \hline \hline 
        Phosphorous (P) & \num{0.08}  & \num{2167} \\
        Gallium (Ga) & \num{300.00}  & \num{1544} \\
        Germanium (Ge) & \num{745.00} & \num{6145} \\
        \hline \hline 
        Platinum (Pt) & \num{12394.00} & \num{1705} \\
        Gold (Au) & \num{12346.00} & \num{49} \\
        Osmium (Os) & \num{12860.00} & \num{307} \\
        \hline
    \end{tabular}
\end{center}

\end{frame}


\begin{frame} %-----------------------------%
\frametitle{Low-thrust vehicles} % electric propulsion
\begin{itemize}
    \item Low-thrust orbital transfers
    \begin{itemize}
        \item Electric propulsion has increased in popularity

        \includegraphics[height=0.3\textheight]{patriot_plume}
        \hfill
        \includegraphics[height=0.3\textheight]{deepspace1}
 
        \item Offers much higher specific impulse than chemical engines 
        \item Requires much longer operating periods for maneuvers 
        \item Small satellites with electric propulsion allows for new mission types
            \begin{itemize}
                \item Formation flight (distributed aperture sensing)
                \item On-orbit servicing
                \item Interplanetary swarms
            \end{itemize}
    \end{itemize}
\end{itemize}
\end{frame}   %-----------------------------%

\section*{}
\subsection*{Previous Work}

\begin{frame}{Asteroid Missions} %-----------------------------%
Use a spherical harmonic model to represent gravity field

Much analysis into the structure of dynamics

Hovering type control
\end{frame}   %-----------------------------%

\section*{}
\subsection*{System Model}

\begin{frame}{Asteroid Gravitational Model}

\end{frame}

\begin{frame}{Equations of Motion}

\end{frame}

\section*{}
\subsection*{Proposed Approach}

\begin{frame}{Reachability Set} % -----------------------------------%

Description of reachability set

Introduce \Poincare section

Generate reachability set on \Poincare section

Use low thrust to enlarge reachability set
\end{frame} %--------------------------------------%

\section*{}
\subsection*{Numerical Simulation}

\begin{frame}{Numerical Simulation} %-----------------------------%

Describe transfer example

Show animation in both body and inertial frame

Show reachability set

% \animategraphics[autoplay,loop,width=0.5\textwidth]{8}{./animation/single_noavoid/single_noavoid-}{0}{99}~
% \animategraphics[autoplay,loop,width=0.5\textwidth]{8}{./animation/single_avoid/single_avoid-}{0}{99}

\end{frame}%-----------------------------%


\section*{}
\subsection*{}

\begin{frame}{Conclusions} %-----------------------------%

	

\end{frame}   %-----------------------------%

\begin{frame}[c]{Thank you}
	\centering
	
	\textbf{\large Flight Dynamics \& Control Lab} \\
	Mechanical \& Aerospace Engineering \\
	School of Engineering \& Applied Science
	
	\begin{figure} %figure%
       	\includegraphics[width=0.75\textwidth]{gw_txh_2cs_pos}
  	\end{figure}
	
	\url{https://fdcl.seas.gwu.edu}
\end{frame}

\end{document}

