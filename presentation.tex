%\documentclass[11pt,professionalfonts,hyperref={pdftex,pdfpagemode=none,pdfstartview=FitH}]{beamer}
%\usepackage{times}
\documentclass[11pt,professionalfonts]{beamer}
\usefonttheme{serif}

\usepackage{aas_presentation_packages}
\bibliography{library} % must be in the preamble when using biblatex package

\DeclareSIUnit\year{yr}

\newcommand{\vs}{\vspace{0.3cm}}

\definecolor{mygray}{gray}{0.9}
\definecolor{RoyalBlue}{rgb}{0.25,0.41,0.88}
\def\Emph{\textcolor{RoyalBlue}}

\definecolor{tmp}{rgb}{0.804,0.941,1.0}
\setbeamercolor{numerical}{fg=black,bg=tmp}
\setbeamercolor{exact}{fg=black,bg=red}

\mode<presentation> 
{
  \usetheme{Warsaw}
  \usefonttheme{serif}
  \setbeamercovered{transparent}
}

\setbeamertemplate{footline}%{split theme}
{%
  \leavevmode%
  \hbox{\begin{beamercolorbox}[wd=.5\paperwidth,ht=2.5ex,dp=1.125ex,leftskip=.3cm,rightskip=.3cm plus1fill]{author in head/foot}%
    \usebeamerfont{author in head/foot}\insertshorttitle
  \end{beamercolorbox}%
  \begin{beamercolorbox}[wd=.5\paperwidth,ht=2.5ex,dp=1.125ex,leftskip=.3cm,rightskip=.3cm]{title in head/foot}
%    \usebeamerfont{title in head/foot}\mypaper\hfill \insertframenumber/\inserttotalframenumber
    \usebeamerfont{title in head/foot}\hfill \insertframenumber/\inserttotalframenumber
  \end{beamercolorbox}}%
  \vskip0pt%
} \setbeamercolor{box}{fg=black,bg=yellow}


\title[Reachability Sets on \Poincare section]{\large\bf  Low-Thrust Trajectory Design Using Reachability Sets near Asteroid 4769 Castalia}

\author{\vspace*{-0.3cm}}

   
\institute{
  \footnotesize
  {\normalsize\bf{Shankar Kulumani and Taeyoung Lee}}\\
  \vspace*{0.2cm}
    \textbf{Flight Dynamics \& Control Lab}\\ \vspace*{0.5cm}
  \begin{figure} %figure%
        \includegraphics[width=0.75\textwidth]{gw_txh_2cs_pos}
    \end{figure}
}
\date{}

\begin{document}
%=======================================================%

\setcounter{framenumber}{-1}
\begin{frame} %-----------------------------%
  \titlepage
\end{frame}   %-----------------------------%

\section*{}
\subsection*{Introduction}  

\begin{frame}{Asteroid Missions}
\begin{itemize}
    \item Science - insight into the early formation of the solar system
    \item Mining - vast quantities of useful materials
    \item Impact - high risk from hazardous near-Earth asteroids
\end{itemize}    

\pause

\begin{center}
    \includegraphics[height=0.35\textheight]{figures/near_mos_20001203_full.jpg}
    \hfill
    \includegraphics[height=0.35\textheight]{figures/Itokawa8_hayabusa_1210.jpg}
\end{center}
\end{frame}

\begin{frame}{Asteroid Mining}
    \begin{itemize}
      \item Useful materials can be extracted from asteroids to support:
      \begin{itemize}
          \item Propulsion, construction, life support, agriculture, and precious/strategic metals
      \end{itemize}
      \item Commercialization of near-Earth asteroids is feasible~\footfullcite{ross2001}
    \end{itemize}

\pause

\begin{center}
\small
    \begin{tabular}{|l|r|r|}
        \hline 
        Element & Price (\SI{}{\$\per\kilo\gram}) & Sales (\SI{}{\$M\per\year}) \\
        \hline \hline 
        Phosphorous (P) & \num{0.08}  & \num{2167} \\
        Gallium (Ga) & \num{300.00}  & \num{1544} \\
        Germanium (Ge) & \num{745.00} & \num{6145} \\
        \hline \hline 
        Platinum (Pt) & \num{12394.00} & \num{1705} \\
        Gold (Au) & \num{12346.00} & \num{49} \\
        Osmium (Os) & \num{12860.00} & \num{307} \\
        \hline
    \end{tabular}
\end{center}

\end{frame}


\begin{frame} %-----------------------------%
\frametitle{Low-thrust vehicles} % electric propulsion
\begin{itemize}
    \item Low-thrust orbital transfers offer increased mission oportunities
    \begin{itemize}
        \item Electric propulsion is increasing in capability
        \item Offers much higher specific impulse than chemical engines 
        \item Requires much longer operating periods for maneuvers 
        \item Enables long duration missions with frequent thrusting
    \end{itemize}
\end{itemize}

\begin{center}
    \includegraphics[height=0.3\textheight]{patriot_plume}
    ~
    \includegraphics[height=0.3\textheight]{deepspace1}
\end{center}
\end{frame}   %-----------------------------%

\section*{}
\subsection*{Past Challenges}

\begin{frame}{Gravitational Modeling} %-----------------------------%

\begin{itemize}
  \item Asteroids are extended bodies not point masses - requires an accurate model
  \pause
  \item Spherical Harmonic - popular, but only valid outside of circumscribing sphere
    {
    \small
    \begin{align*}
      U(\vecbf{r} ) = \frac{\mu}{r} \sum_{n=0}^\infty \sum_{m=0}^\infty \parenth{\frac{R}{r}}^nP_{n,m}(\sin \phi) \braces{ C_{nm} \cos(m \lambda) + S_{nm} \sin(m \lambda)} 
    \end{align*}
    }
    \pause
  \item Infinite series is always an approximation and adds complexity
    \begin{itemize}
        \item Model switching at circumscribing sphere
        \item Coefficient matching to ensure continuity across switching region
    \end{itemize}
\end{itemize}

\note[itemize]{
  \item Models require detailed data from orbit about asteroid (OD process determines gravity field)
  \item Simplified models (triaxial ellipsoid allows analytical insight)
  \item Previous work fails to consider coupled dyanmics
  }

\end{frame}   %-----------------------------%

\begin{frame}{Optimal Transfers} %-----------------------------%

\begin{itemize}
    \item Optimal Trajectory Design
        \begin{itemize}
            \item Orbital dynamics are nonlinear and chaotic
            \item Very sensitive to initial conditions
            \item Intuition required by designer to enable convergence
        \end{itemize}
    \pause
    \item Direct Optimal Control
        \begin{itemize}
            \item Reformulate problem as parameter optimization
            \item Allows for use of nonlinear programming methods
            \item High dimensional problem and computationally intensive
            \item Results in suboptimal solutions due to discretization
        \end{itemize}
\end{itemize}
\end{frame}   %-----------------------------%

\section*{}
\subsection*{System Model}

\begin{frame}{Polyhedron Gravitation Model}

\begin{itemize}
    \item Potential is a function of only the shape model
    \item Globally valid, closed-form expression of potential
    \item Exact potential assumes a constant density and an accurate shape model
\end{itemize}
\only<2>{
\begin{align*}\label{eq:potential}
    U(\vecbf{r}) &= \frac{1}{2} G \sigma \sum_{e \in \text{edges}} \vecbf{r}_e \cdot \vecbf{E}_e \cdot \vecbf{r}_e \cdot L_e - \frac{1}{2}G \sigma \sum_{f \in \text{faces}} \vecbf{r}_f \cdot \vecbf{F}_f \cdot \vecbf{r}_f \cdot \omega_f \in \R^1
\end{align*}
}
\only<3>{
\begin{center}
  \animategraphics[autoplay,loop,width=0.5\textwidth]{30}{./animation/castalia/IMG}{00001}{00999}~\hfill
  \includegraphics[width=0.5\textwidth]{figures/radius_contour.pdf}
\end{center}
}

\end{frame}

\begin{frame}{Equations of Motion}

\begin{itemize}
    \item Many similarities to the three-body problem
    \item Equations are also defined in body fixed frame
\end{itemize}

\[
    \begin{bmatrix} \dot{\vecbf{r}} \\ \dot{\vecbf{v}} \end{bmatrix} =
    \begin{bmatrix}\vecbf{v} \\ \vecbf{g} \parenth{\vecbf{r}} + \vecbf{h}\parenth{\vecbf{v}} + \vecbf{u} \end{bmatrix} ,
\]
\pause
\begin{itemize}
    \item Dynamics allow for a single integral of motion 
    \item Critical to defining periodic solutions via differential corrections
\end{itemize}

\[
    J \parenth{\vecbf{r}, \vecbf{v}} = \frac{1}{2} \omega^2 \parenth{x^2 + y^2} + U(\vecbf{r}) - \frac{1}{2} \parenth{\dot{x}^2 + \dot{y}^2 + \dot{z}^2} .
\]
\end{frame}

\section*{}
\subsection*{Proposed Approach}

\begin{frame}{Proposed Approach} % -----------------------------------%
  \begin{itemize}
      \item We use the \Emph{reachability set} to design transfers
      \item Multiple iterations of the reachable set are used to achieve large/complex transfers
      \item Design transfers using a lower dimensional \Poincare section 
      \item \Emph{Reachability set} on Poincar\'e section allows for systematic transfer design
  \end{itemize}

  \note[itemize]{
    \item Reachability set avoids the need to pick initial conditions
    \item We compute on a lower dimensional surface
  }
\end{frame} %--------------------------------------%

\begin{frame}{\Poincare map}
\begin{itemize}
    \item Intersection of a periodic orbit with a lower dimensional subspace, called the \Poincare section
    \pause
        \begin{itemize}
            \item Can be considered a discrete map between intersections of the section
        \end{itemize}
        \pause
    \item Useful for investigating the stability and structure of dynamical systems
    \pause
    \item We define a section \( \Sigma \) to determine initial and target periodic orbits for our transfer
\end{itemize}

\begin{align}\label{eq:poincare_section}
    \Sigma = \braces{\parenth{x, \dot{x}, z, \dot{z}} | y(t_f) = 0 }.
\end{align}

\end{frame}

\begin{frame}{Reachability Set}

\begin{itemize}
    \item Set of states achievable from a given initial condition over fixed \( t_f \) s.t. maximum control contraint
    \[
    R( \vecbf{x}_0, \mathcal{U} , t_f) = \braces{ \vecbf{x}_f \subseteq \mathcal{X} | \exists \vecbf{u} \in \mathcal{U}, \vecbf{x}(t_f) = \vecbf{x}_f }
    \]
    \pause
    \item Directly derivable from optimal control
    \item Frequently used for safety planning, e.g. air traffic collision avoidance
    \pause
    \item We extend its use to the design of spacecraft transfers
\end{itemize}

\end{frame}

\begin{frame}{Reachability Set on \Poincare section} % -----------------------------------%

\begin{itemize}
    \item We seek to generate the reachability set on a chosen \Poincare section
    \[
        \Sigma = \braces{\parenth{x, \dot{x}, z, \dot{z}} | y(t_f) = 0 }.
    \]
    \item Control input is chosen to enlarge the reachable set
\end{itemize}
\pause
\begin{figure}
    \centering
    \begin{scaletikzpicturetowidth}{0.3\textwidth}
    \begin{tikzpicture}[scale=\tikzscale]
        \coordinate [label=left:\textcolor{black}{\large \(\vecbf{x}_0\)}] (x0) at (-1,-2);
        \coordinate [label=below:\textcolor{black}{\large  \(\vecbf{x}_n\)}] (xn) at (1,1);
        \coordinate [label=left:\textcolor{black}{\large  \(\Sigma\)}] (sigma) at (-4,3);
        %\coordinate [label=below:\textcolor{black}{\large  \(P(\vec{x})\)}] (P) at (0,-3.5);
        % define the path of the flow with coordinates
        \coordinate [label=right:\textcolor{black}{}] (f1) at (5,-2);
        \coordinate [label=below:\textcolor{black}{\large  \(\psi(t,\vecbf{x}_0)\)}] (f2) at (2,-5);
        \coordinate [label=right:\textcolor{black}{}] (f3) at (-4,-4);
        \coordinate [label=right:\textcolor{black}{}] (f4) at (-4,-1);
        
    %   \draw[help lines] (-10,-10) grid (10,10); %grid
        \filldraw [black] (x0) circle [radius=3pt];
        \filldraw [black] (xn) circle [radius=3pt];
    
        \draw [ultra thick,black,->-](x0) to[out=20,in=90,distance=2cm] (f1) to[out=-90,in=0,distance=2cm] (f2) to[out=180,in=-45,distance=2cm] (f3) to[out=135,in=-135,distance=2cm] (f4) ;
        \draw [ultra thick, black,dashed,->] (f4) to[out=45,in=180,distance=1cm] ($(xn)-(2,0)$);
        
        \draw [ultra thick] plot [smooth cycle, tension=0.1, rotate=5] coordinates { (-4,-3) (4,-3) (4,3) (-4,3) };
    
        \draw [thick,dashed] (xn) circle [radius=2cm]; % reachability set
    
        \draw [thick,->] (xn) -- ($(xn) + (2.5,0)$);
        \draw [thick,rotate=45,->] (xn) -- ($(xn) + (2.5,0)$);
        \draw ($(xn) + (1,0)$) arc [start angle=0,end angle=45, radius=1];
        \node [draw=none] at (2.4,1.5) {\Large \(\phi_d\)};
        \draw [decorate,decoration={brace,amplitude=5pt},rotate=45] (xn) -- ($(xn) + (2,0)$);
        \node [draw=none] at ($ (xn) + (0,1) $) {\Large \( J \)};
    \end{tikzpicture}
    \end{scaletikzpicturetowidth}
\end{figure}

\end{frame} %--------------------------------------%

\begin{frame}{Optimal Control Problem}
\begin{itemize}
    \item Reachability defined as distance between controlled and uncontrolled states
    {\small
        \[
            J = -\frac{1}{2} \left( \vecbf{x}(t_f) - \vecbf{x}_{n}(t_f)\right)^T 
            Q
            \left( \vecbf{x}(t_f) - \vecbf{x}_{n}(t_f)\right) ,
        \]
    }
    \pause
    \item Terminal constraints used to ensure correct section and specific direction on \( \Sigma \in \R^4 \)
    {\small
        \begin{align*}\label{eq:terminal_constraints}
            \begin{split}
                m_1 &= y = 0 , \\
                m_2 &= \parenth{\sin \phi_{1_{d}}} \parenth{ x_1^2 + x_2^2 + x_3^2 + x_4^2} - x_1^2 = 0, \\
                m_3 &= \parenth{\sin \phi_{2_{d}}} \parenth{ x_2^2 + x_3^2 + x_4^2} - x_2^2 = 0, \\
                m_4 &= \parenth{\sin \phi_{3_{d}}} \parenth{ 2 x_3^2 + 2 x_3 \sqrt{x_4^2 + 2 x_4^2}} - x_3 - \sqrt{x_4^2 + x_3^2} = 0 ,
            \end{split}
        \end{align*}
    }
    \pause
    \item Control constraint used to emulate realistic system
        {\small
        \[
            c(\vecbf{u}) = \vecbf{u}^T \vecbf{u} - u_m^2 \leq 0 ,
        \]
        }
\end{itemize}

\end{frame}

\begin{frame}{Two Point Boundary Value Problem}
\begin{itemize}
    \item Multiple shooting used to solve necessary conditions
    \pause
    \item Approximate the reachable set via \( \phi_1, \phi_2, \phi_3 \) 
    \pause
    \item From the reachable set we chose the state which minimizes \( d \) to the target
    \item Compute another reachable set if target is outside of set
\end{itemize}

\begin{align}\label{eq:reach_dist}
    d = \sqrt{k_x \parenth{x_f - x_t }^2 + k_z \parenth{z_f - z_t }^2 + k_{\dot{x}}\parenth{\dot{x}_f - \dot{x}_t }^2 + k_{\dot{z}}\parenth{\dot{z}_f - \dot{z}_t }^2} ,
\end{align}

\end{frame}

\section*{}
\subsection*{Numerical Simulation}

\begin{frame}{Transfer Objective} %-----------------------------%

\begin{itemize}
    \item Goal is to transfer between two equatorial periodic orbits
    \item Typical scenario during study of an asteroid
\end{itemize}

\begin{center}
    \includegraphics[width=0.5\textwidth]{figures/initial_transfer.pdf}
    \hfill
    \includegraphics[width=0.5\textwidth]{figures/initial_transfer_3d.pdf}
\end{center}

\end{frame}%-----------------------------%

\begin{frame}{Simulation}


\begin{itemize}
    \item Generate the reachability set through discretization of \( \phi_i \)
    \item Visualize \( \Sigma \in \R^4 \) through the use of two 2-D sections
    \pause
    \item Control input allows for large deviation in velocity components
\end{itemize}

\begin{center}
    \includegraphics[width=0.5\textwidth]{figures/poincare_xvsxdot.pdf}
    \hfill
    \includegraphics[width=0.5\textwidth]{figures/poincare_zvszdot.pdf}
\end{center}

\end{frame}

\begin{frame}{Simulation}
    \begin{itemize}
        \item Four iterations of the reachable state to meet the target set
        \item Final transfer is computed with a fixed terminal state constraint
    \end{itemize}

    \begin{center}
        \includegraphics[width=0.5\textwidth]{figures/trajectory.pdf}
        \hfill
        \includegraphics[width=0.5\textwidth]{figures/trajectory_3d.pdf}
    \end{center}

\end{frame}

\begin{frame}{Complete transfer}
\begin{itemize}
    \item We can visualize the complete trajectory in both the body and inertial frames
\end{itemize}

\begin{center}
  \animategraphics[autoplay,loop,width=0.5\textwidth]{30}{./animation/body/IMG}{00001}{01499}~\hfill
  \animategraphics[autoplay,loop,width=0.5\textwidth]{30}{./animation/inertial/IMG}{00001}{01499}
\end{center}

\end{frame}

\section*{}
\subsection*{}

\begin{frame}{Conclusions} %-----------------------------%
\begin{itemize}
    \item Challenging transfer utilizing multiple iterations of the reachability set
    \item Alleviates the need for selecting accurate initial guesses for convergence
    \item Gives insight into the possible motion of the spacecraft
    \item Extension of previous work in the planar three-body problem
\end{itemize}

\end{frame}   %-----------------------------%

\begin{frame}[c]{Thank you}
  \centering
  
  \textbf{\large Flight Dynamics \& Control Lab} \\
  Mechanical \& Aerospace Engineering \\
  School of Engineering \& Applied Science
  
  \begin{figure} %figure%
        \includegraphics[width=0.75\textwidth]{gw_txh_2cs_pos}
    \end{figure}
  
  \url{https://fdcl.seas.gwu.edu}
\end{frame}

\end{document}

